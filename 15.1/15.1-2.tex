\documentclass{article}
\usepackage{amsmath,url}
\title{CLRS 15.1-4}
\author{Peter Danenberg}
\begin{document}
\maketitle
Thanks, \r{A}smund Eldhuset \cite{eldhuset:07}; $r_i$ can be
generalized followingly:

\begin{align}
  r_i(n) = r_1(n) = r_2(n) && \text{(15.1), (15.8)}\label{general}
\end{align}

Similarly:

\begin{align}
  r_i(n) = r_1(j+1)+r_2(j+1) = 2r_i(j+1) && \text{(\ref{general}),
    (15.9)}\label{extension}
\end{align}

Thus:

\begin{align}
  r_1(n) = r_2(n) = r_i(n) = 2^{n-n} = 1 && \text{(15.8), (15.1-4)}
\end{align}

Lastly:

\begin{align}
  r_i(j) &= 2r_i(j+1) & \text{(\ref{extension})}\label{recapit} \\
  &= 2\cdot 2^{n-(j+1)} & \text{(\ref{recapit}), (15.1-4)} \\
  &= 2^{1+n-j-1} \\
  &= 2^{n-j} \\
\end{align}
\bibliographystyle{plain}
\bibliography{15.1-2}
\end{document}
